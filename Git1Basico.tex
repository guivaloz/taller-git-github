%
% Taller de Git y GitHub
%
% Comando Git parte 1
%

\section{Git básico}

\subsection{Primeros pasos}
\begin{frame}
  \frametitle{Versión instalada}
  Verifique que tenga instalado Git.
  \begin{block}<1->{Para averiguar la versión instalada}
    \$ git version\par
    git version 2.4.6
  \end{block}
  \begin{block}<2->{Para instalar en la famila de Debian}
    \$ sudo apt-get install git-cvs
  \end{block}
  \begin{block}<3->{Ejecute Git sin parámetros para mostrar una ayuda simple}
    \$ git
  \end{block}
\end{frame}

\subsection{Configurar}
\begin{frame}
  \frametitle{Configuraciones globales}
  Es obligado que configure su nombre e e-mail.
  \begin{block}<1->{Definir su nombre y correo electrónico}
    \$ git config --global user.name "Tu Nombre Completo"\par
    \$ git config --global user.email tunombre@servidor.com
  \end{block}
  \begin{block}<2->{Usar colores en la terminal}
    \$ git config --global color.ui auto
  \end{block}
  \begin{block}<3->{Generar un par de llaves OpenSSH}
    \$ ssh-keygen
  \end{block}
\end{frame}

\subsection{Iniciar repositorio}
\begin{frame}
  \frametitle{Iniciar un repositorio}
  Para iniciar un repositorio cámbiese al directorio base del mismo y ejecute git init.
  \begin{block}<1->{Crear su Primer Repositorio}
    \$ cd \textasciitilde/Documentos/GitHub/PrimerRepositorio\par
    \$ git init
  \end{block}
  Todos los comandos Git de este repositorio debe ejecutarlos en este directorio. Se creará un directorio oculto con nombre .git.
  \begin{block}<2->{Configure lo que NO se compartirá}
    \$ nano .git/info/exclude
  \end{block}
\end{frame}

\subsection{Agregar novedades}
\begin{frame}
  \frametitle{Agregar novedades al repositorio}
  Haga cambios en los archivos o cree nuevos.
  \begin{block}<1->{Revisar el status}
    \$ cd \textasciitilde/Documentos/Prueba\par
    \$ git status
  \end{block}
  \begin{block}<2->{Agregar archivos al repositorio local}
    \$ git add .
  \end{block}
  \begin{block}<3->{Hacer un corte: es su respaldo y lo deja listo para subir}
    \$ git commit -m "He hecho unas mejoras para aprender."
  \end{block}
  \begin{block}<4->{Revisar la bitácora}
    \$ git log
  \end{block}
\end{frame}
