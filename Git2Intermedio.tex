%
% Taller de Git y GitHub
%
% Comando Git parte 2
%

\section{Git Intermedio}

\subsection{Ramas}
\begin{frame}
  \frametitle{Ramas}
  Un \textit{branch} es una \textbf{rama} que permite establecer una ruta distinta; que más adelante podría integrarse a la \textbf{rama principal}.
  \begin{block}<1->{Mostrar la rama en uso}
    \$ git branch\par
  \end{block}
  \begin{block}<2->{Agregar una nueva rama y cambiarse a ésta}
    \$ git branch guillermo\par
    \$ git checkout guillermo
  \end{block}
  \begin{block}<3->{Revise las ramas y sus últimos comentarios}
    \$ git branch -v
  \end{block}
\end{frame}

\subsection{Cómo funcionan las ramas}
\begin{frame}
  \frametitle{Cómo funcionan las ramas}
  Si elimina o renombra archivos o directorios deberá usar el parámetro \textbf{--all} al agregar.
  \begin{block}<1->{Agregue novedades y haga un corte}
    \$ git status\par
    \$ git add . --all\par
    \$ git status\par
    \$ git commit -m "He hecho un par de mejoras."\par
  \end{block}
  \begin{block}<2->{Cámbiese a la rama master, queda como antes de sus cambios}
    \$ git checkout master
  \end{block}
  \begin{block}<3->{Regrese a su rama, regresarán los cambios}
    \$ git checkout guillermo
  \end{block}
\end{frame}

\subsection{Fusionar una rama}
\begin{frame}
  \frametitle{Fusionar una rama}
  Lo más sano es ir integrando novedades a las ramas. Cuando queden listas, se integran a la rama \textbf{master}.
  Lo que es lo mismo, mantenga \textbf{master} \textit{atrás} como lo estable y a las ramas \textit{adelante} con las novedades.
  \begin{block}<1->{Fusionar la rama guillermo en master}
    \$ git checkout master\par
    \$ git merge guillermo
  \end{block}
  \begin{block}<2->{Verifique que master y la rama guillermo son iguales}
    \$ git branch -v
  \end{block}
  \begin{block}<3->{Cuando ya no la necesite; puede eliminar una rama}
    \$ git branch -d guillermo
  \end{block}
\end{frame}

\subsection{Subir una rama a GitHub}
\begin{frame}
  \frametitle{Subir una rama a GitHub}
  Cuando tenga avances terminados o \textit{commiteados} que compartir con sus colegas en GitHub.
  \begin{block}<1->{Subir una rama a GitHub}
    \$ git push origin nombredelarama
  \end{block}
  \begin{block}<2->{Sus amigos pueden bajar su rama}
    \$ git fetch origin nombredelarama\par
    \$ git checkout nombredelarama\par
    \$ git pull origin nombredelarama
  \end{block}
\end{frame}

\subsection{Más órdenes útiles}
\begin{frame}
  \frametitle{Más órdenes útiles}
  No hay mejor forma de aprender que usándolo.
  \begin{block}<1->{Sincronice su copia local y lea los cambios de las ramas}
    \$ git fetch\par
    \$ git branch -v
  \end{block}
  \begin{block}<2->{Para regresar al pasado, destruyendo lo nuevo}
    \$ git reset --hard 1234567
  \end{block}
\end{frame}
